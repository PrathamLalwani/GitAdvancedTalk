
\documentclass{beamer}

% Use the Copenhagen theme for an academic and clean look
\usetheme{Copenhagen}

% Packages for code listings, graphics, and TikZ illustrations
\usepackage{listings} % for code snippets
\usepackage{graphicx} % for including images
\usepackage{tikz}     % for diagrams
\usepackage{hyperref} % for clickable links

% Set title, subtitle, author, institute and date information
\title[Advanced Git]{Advanced Git and Bash Techniques for Software Development}
\subtitle{An In-depth Workshop for Undergraduate Students}
\author{Your Name}
\institute[Your Institution]{Your Institution, Department of Computer Science}
\date{\today}

% Custom code listing settings
\lstset{
  basicstyle=\ttfamily\footnotesize,
  breaklines=true,
  numbers=left,
  numberstyle=\tiny,
  frame=single,
  captionpos=b
}

\begin{document}

%%%%%%%%%%%%%%%%%%%%%%%%%%%%%%%%%%%%%%%%%%%%%%%%%%%%%%%%%%%%%%%%%%%%%%%%%%%%%%%
% Title Slide
\begin{frame}
	\titlepage
\end{frame}

%%%%%%%%%%%%%%%%%%%%%%%%%%%%%%%%%%%%%%%%%%%%%%%%%%%%%%%%%%%%%%%%%%%%%%%%%%%%%%%
% Table of Contents
\begin{frame}
	\frametitle{Outline}
	\tableofcontents
\end{frame}

%%%%%%%%%%%%%%%%%%%%%%%%%%%%%%%%%%%%%%%%%%%%%%%%%%%%%%%%%%%%%%%%%%%%%%%%%%%%%%%
% Section 1: Review Git Basics
\section{Review Git Basics}
% Slide 1.1: Introduction to Basic Commands
\begin{frame}[fragile]
	\frametitle{Review Git Basics: Introduction}
	In this section we review the core Git commands and operations that every developer should know. \pause
	\begin{itemize}
		\item \textbf{\texttt{git init}}: Initialize a new repository.
		\item \textbf{\texttt{git add}}: Stage changes for commit.
		\item \textbf{\texttt{git commit}}: Commit the staged changes.
		\item \textbf{\texttt{git push}}: Push the commits to a remote repository.
	\end{itemize}
	\vfill
	\textbf{Time: 10 minutes (demo and discussion)}
\end{frame}

% Slide 1.2: Live Demo Example: Repository Initialization and Simple Workflow
\begin{frame}[fragile]
	\frametitle{Git Workflow Example}
	\textbf{Example Workflow:}
	\begin{lstlisting}[language=bash]
  mkdir my_project
  cd my_project
  git init
  echo "Hello, Git!" > README.md
  git add README.md
  git commit -m "Initial commit"
  git remote add origin https://github.com/username/my_project.git
  git push -u origin master
  \end{lstlisting}
	\vfill
	\alert{Demonstration:} Walk through these commands live or use a recorded terminal session.
\end{frame}

%%%%%%%%%%%%%%%%%%%%%%%%%%%%%%%%%%%%%%%%%%%%%%%%%%%%%%%%%%%%%%%%%%%%%%%%%%%%%%%
% Section 2: Useful Bash Commands
\section{Some Bash Commands}
% Slide 2.1: Introduction to Bash Utilities
\begin{frame}
	\frametitle{Enhancing Your Shell Workflow}
	In this section, we cover some handy Bash tools to improve productivity:
	\begin{itemize}
		\item \texttt{tldr}: Concise command help.
		\item \texttt{fzf}: Fuzzy finder for quick navigation.
		\item \texttt{grep}: Powerful text search.
		\item \texttt{cd aliases}: Speed up directory navigation.
		\item \texttt{find}: Locate files by pattern.
	\end{itemize}
	\vfill
	\textbf{Time: 15 minutes (interactive session and examples)}
\end{frame}

% Slide 2.2: Examples of Bash Commands
\begin{frame}[fragile]
	\frametitle{Bash Commands in Action}
	\textbf{\texttt{tldr} and \texttt{fzf}:}
	\begin{lstlisting}[language=bash]
  tldr tar
  fzf --reverse
  \end{lstlisting}
	\medskip
	\textbf{Custom Alias Example:}
	\begin{lstlisting}[language=bash]
  alias proj='cd ~/projects'
  \end{lstlisting}
	\vfill
	\uncover<2->{Try these commands in your terminal after the demonstration.}
\end{frame}

% Slide 2.3: Combining Commands for Productivity
\begin{frame}[fragile]
	\frametitle{Combining Commands in Scripts}
	\textbf{Using \texttt{grep} and \texttt{find}:}
	\begin{lstlisting}[language=bash]
  find . -name "*.log" | xargs grep -i "error"
  \end{lstlisting}
	This command searches for “error” in all \texttt{.log} files recursively in the current directory.
	\vfill
	\pause
	\textbf{Discussion:} Share tips on how these commands can be integrated into daily workflows.
\end{frame}

%%%%%%%%%%%%%%%%%%%%%%%%%%%%%%%%%%%%%%%%%%%%%%%%%%%%%%%%%%%%%%%%%%%%%%%%%%%%%%%
% Section 3: Advanced Git Techniques
\section{Advanced Git Techniques}
% Slide 3.1: Git Log and Viewing History
\begin{frame}[fragile]
	\frametitle{Advanced Git: Exploring History}
	\textbf{Git Log Options:}
	\begin{itemize}
		\item \texttt{git log --oneline} for a compact view.
		\item \texttt{git log --graph --decorate} to see branch structure.
	\end{itemize}
	\vfill
	\lstset{language=bash}
	\begin{lstlisting}
  git log --oneline --graph --decorate
  \end{lstlisting}
	\vfill
	\pause
	Use these commands to visually inspect the commit history and branch merges.
\end{frame}

% Slide 3.2: Branching, Merging and Conflict Resolution
\begin{frame}[fragile]
	\frametitle{Branching and Merging Strategies}
	\textbf{Key Concepts:}
	\begin{itemize}
		\item \textbf{Branches:} Create feature branches using \texttt{git checkout -b}.
		\item \textbf{Merging:} Use \texttt{git merge} to integrate changes.
		\item \textbf{Conflict Resolution:} Use diff and merge tools to resolve conflicts.
	\end{itemize}
	\vfill
	\textbf{Example:}
	\begin{lstlisting}[language=bash]
  git checkout -b feature/new-feature
  # Make changes and commit
  git checkout master
  git merge feature/new-feature
  \end{lstlisting}
	\vfill
	\alert{Note:} Explain fast-forward merges versus recursive merges.
\end{frame}

% Slide 3.3: Squashing Commits and Reverting Changes
\begin{frame}[fragile]
	\frametitle{Cleaning Up History: Squash and Revert}
	\textbf{Squash Commits:}
	\begin{lstlisting}[language=bash]
  git rebase -i HEAD~3
  \end{lstlisting}
	Use interactive rebase to combine several commits into one for a cleaner history.
	\vfill
	\textbf{Reverting Changes:}
	\begin{lstlisting}[language=bash]
  git revert <commit-hash>
  \end{lstlisting}
	This command safely undoes a specific commit without rewriting history.
	\vfill
	\pause
	\textbf{Discussion:} When is it more appropriate to revert versus perform a rebase?
\end{frame}

%%%%%%%%%%%%%%%%%%%%%%%%%%%%%%%%%%%%%%%%%%%%%%%%%%%%%%%%%%%%%%%%%%%%%%%%%%%%%%%
% Section 4: Contributing to Open Source
\section{Contributing to Open Source}
% Slide 4.1: Overview of the Open Source Contribution Workflow
\begin{frame}
	\frametitle{Contributing to Open Source: Getting Started}
	Key Steps:
	\begin{enumerate}
		\item \textbf{Forking:} Create your own fork of the repository.
		\item \textbf{Cloning:} Clone your fork locally using \texttt{git clone}.
		\item \textbf{Feature Branch:} Create a branch for your changes.
		\item \textbf{Making Changes:} Code improvements, fixes, or documentation edits.
		\item \textbf{Pull Request:} Submit your changes for review via a PR.
	\end{enumerate}
	\vfill
	\pause
	\alert{Tip:} Use GitHub’s GUI and command line tools to manage this workflow.
\end{frame}

% Slide 4.2: Detailed Workflow with Commands
\begin{frame}[fragile]
	\frametitle{Detailed Contribution Workflow}
	\textbf{Commands:}
	\begin{lstlisting}[language=bash]
  # Fork on GitHub, then clone your fork:
  git clone https://github.com/yourusername/repository.git
  cd repository
  git checkout -b feature-improvement
  
  # Make changes, then commit:
  git add .
  git commit -m "Improve feature X"
  
  # Push your branch:
  git push origin feature-improvement
  \end{lstlisting}
	\vfill
	Then, open GitHub and create a Pull Request. Discuss best practices and code review tips.
\end{frame}

%%%%%%%%%%%%%%%%%%%%%%%%%%%%%%%%%%%%%%%%%%%%%%%%%%%%%%%%%%%%%%%%%%%%%%%%%%%%%%%
% Section 5: Additional Topics and Q&A
\section{Additional Topics and Q\&A}
% Slide 5.1: Common Pitfalls and Troubleshooting
\begin{frame}[fragile]
	\frametitle{Common Pitfalls \& Troubleshooting in Git}
	\begin{block}{Tips}
		\begin{itemize}
			\item Always backup your work before performing a rebase.
			\item Use \texttt{git status} and \texttt{git diff} frequently.
			\item Read error messages carefully and use online searches (e.g., Stack Overflow).
		\end{itemize}
	\end{block}
	\vfill
	\pause
	\textbf{Interactive:} Invite questions on real-world issues students have encountered.
\end{frame}

% Slide 5.2: Recap and Further Resources
\begin{frame}
	\frametitle{Recap and Further Resources}
	\begin{itemize}
		\item \textbf{Review Git Basics:} Core commands and simple workflows.
		\item \textbf{Bash Tips:} Useful commands to boost shell efficiency.
		\item \textbf{Advanced Git:} Viewing history, branching strategies, squashing, and reverting.
		\item \textbf{Open Source:} Contributing workflow from forking to pull requests.
	\end{itemize}
	\vfill
	Additional reading:
	\begin{itemize}
		\item \href{https://git-scm.com/book/en/v2}{Pro Git Book}
		\item \href{https://www.atlassian.com/git/tutorials}{Atlassian Git Tutorials}
		\item \href{https://www.gnu.org/software/bash/}{GNU Bash Manual}
	\end{itemize}
\end{frame}

%%%%%%%%%%%%%%%%%%%%%%%%%%%%%%%%%%%%%%%%%%%%%%%%%%%%%%%%%%%%%%%%%%%%%%%%%%%%%%%
% Section 6: Appendix
\appendix
\section*{Appendix: Extra Examples and Q\&A}
% Slide A.1: Additional Git Log Overlays
\begin{frame}[fragile]
	\frametitle{Advanced Git Log Visualization}
	\begin{lstlisting}[language=bash]
  git log --oneline --graph --decorate --all
  \end{lstlisting}
	Use this command to view a complete, branch-inclusive visualization of the repository history.
	\vfill
	\pause
	\textbf{Q\&A:} Open the floor for student questions or share additional examples as needed.
\end{frame}

% Slide A.2: Thank You and Contact Information
\begin{frame}
	\frametitle{Thank You!}
	\begin{center}
		Questions?\\[1em]
		\textbf{Your Name}\\[0.5em]
		\texttt{your.email@institution.edu}
	\end{center}
	\vfill
	\hyperlink{contents}{\beamergotobutton{Back to TOC}}
\end{frame}

\end{document}
